\documentclass[a4paper,12pt]{article}
\usepackage[UTF8]{ctex}
\linespread{1.5}%1.5倍行距
\usepackage[top=3.0cm,bottom=3.0cm,left=3.0cm,right=3.0cm]{geometry}%页边距
\usepackage[bookmarks=true,colorlinks=true,linkcolor=blue,urlcolor=green,citecolor=blue,linktocpage=true,hyperindex=true]{hyperref}%使ref,cite有超链接,记得xelatex2遍
\usepackage{amscd}
\usepackage{amsfonts}
\usepackage{amsmath}
\usepackage{amssymb}
\usepackage{amstext}%多种公式环境和数学命令
\usepackage{amsthm}
\usepackage{array}%数组和表格制作
\usepackage{caption}
\usepackage{color}
\usepackage{commutative-diagrams}%用CoDi包画交换图
\usepackage{dsfont}
\usepackage{epsfig}
%\usepackage{epstopdf}%能够插入eps格式图片
\usepackage{fancyhdr}%页眉页脚设置
\usepackage{float}
\usepackage{fontspec}%和字体有关
\usepackage{graphicx}%插图
\usepackage{indentfirst}%首行缩进宏包
\usepackage{latexsym}
\usepackage{listings}
\usepackage{mathrsfs}
\usepackage{multicol}%跨列表格
\usepackage{multirow}%跨行表格
\usepackage[numbers,sort&compress]{natbib}%参考文献,已弃{cite}
\usepackage{rawfonts}
\usepackage{tabularx}%自动设置表格列宽
\usepackage{tikz}
\usepackage{tikz-cd}%tikzcd画交换图
\usepackage{titlesec}%标题设置
\usepackage{titletoc}%目录格式设置
\usepackage{xcolor}
\usepackage[all]{xy}
%--------------------------以下是数学缩写
\newcommand{\A}{\mathscr{A}}
\newcommand{\ad}{\operatorname{ad}}
\newcommand{\Aut}{\operatorname{Aut}}
%\newcommand{\C}{\mathbb{C}}%和hyperref宏包冲突
\renewcommand{\char}{\operatorname{char}}
\newcommand{\Der}{\operatorname{Der}}
\newcommand{\End}{\operatorname{End}}
\newcommand{\Ext}{\operatorname{Ext}}
\newcommand{\F}{\mathcal{F}}
\newcommand{\gl}{\mathrm{gl}}
\newcommand{\grad}{\operatorname{grad}}
\newcommand{\Hom}{\operatorname{Hom}}
\newcommand{\id}{\mathds{1}}
\newcommand{\ima}{\operatorname{im}}
\newcommand{\N}{\mathbb{N}}
\renewcommand{\O}{\mathcal{O}}
\newcommand{\Q}{\mathbb{Q}}
\newcommand{\R}{\mathbb{R}}
\newcommand{\rad}{\operatorname{rad}}
\newcommand{\rank}{\operatorname{rank}}
%默认的\S是2个s构成的花section
\newcommand{\sign}{\operatorname{sign}}
\renewcommand{\sl}{\mathrm{sl}}
\renewcommand{\span}{\operatorname{span}}
\newcommand{\spec}{\operatorname{spec}}
\newcommand{\supp}{\operatorname{supp}}
\newcommand{\Tor}{\operatorname{Tor}}
\newcommand{\tr}{\operatorname{tr}}
\newcommand{\X}{\mathfrak{X}}
\newcommand{\Z}{\mathbb{Z}}
\renewcommand{\L}{\mathcal{L}}
\newcommand{\dif}[2]{\frac{d{#1}}{d{#2}}}
\newcommand{\pa}[2]{\frac{\partial{#1}}{\partial{#2}}}
\newcommand{\abs}[1]{\left\vert#1\right\vert}
\newcommand{\lang}[1]{\left\langle#1\right\rangle}
\newcommand{\norm}[1]{\left\Vert#1\right\Vert}
\newcommand{\paren}[1]{\left(#1\right)}
%------------------------以下是定理排版
\theoremstyle{plain}
%也可按需选择{plain}{definition}
\newtheorem{thm}{Theorem}[section]
\newtheorem{cor}[thm]{Corollary}
\newtheorem{lem}[thm]{Lemma}
\newtheorem{prop}[thm]{Proposition}
\theoremstyle{definition}
\newtheorem{defn}[thm]{Definition}
\newtheorem{rem}[thm]{Remark}
\newtheorem{eg}[thm]{Example}
\newtheorem{ex}[thm]{Exercise}
\renewcommand{\proofname}{Proof}
\renewcommand{\refname}{References}
\renewcommand{\contentsname}{Contents}
\everymath{\displaystyle}%好看不拥挤的数学公式
%-----------------正文开始 qwq --------------
\begin{document}

\title{Article模板}
\author{姓名}
\date{\today}
\maketitle

\tableofcontents

\section{Introduction}
\subsection{Definitions}




$\check{e}\acute{e}\grave{e}\breve{e}\ddot{x}\dot{y}$

















% 长公式,用\[ \begin{aligned} & \\等命令
% 公式标记,形如\eqno(*),放在公式末尾\]之前

% \[
%     \begin{aligned}
%         L_{g}(\gamma) = & \int_0^1 |\gamma'(t)|_g dt                           \\
%         =               & \int_0^1 h(\gamma(t))|\gamma'(t)|_{\widetilde{g}} dt \\
%         \geq            & \int_0^1 \gamma'(f)dt                                \\
%         =               & \int_0^1 \frac{d}{dt}f(\gamma(t))dt                  \\
%         =               & f(\gamma(1))+0+0+0+0+0+0+0                           \\
%                         & -f(\gamma(0))                                        \\
%         =               & f(q)-f(p).
%     \end{aligned}\eqno(*)
% \]

%左大括号,用\left\{  ;然后\begin{aligned} ;右边无配对括号,用\right.

% \[
%     f(x)=\left\{
%     \begin{aligned}
%         & x,  &  & x \in A,    \\
%         & -x, &  & x \notin A.
%     \end{aligned}
%     \right.
% \]

% 矩阵用\left( \begin{array}{cccc表示矩阵的阶数} 然后用&和\\

% \[
%     \det A=\det
%     \left(
%     \begin{array}{cccc}
%             a_{11} & a_{12} & \cdots & a_{1n} \\
%             a_{21} & a_{22} & \cdots & a_{2n} \\
%             \cdots &        & \ddots & \vdots \\
%         \end{array}
%     \right)
% \]


% 脚注用\footnote{出现在本页下方脚注的文字}
% 引用定理等,用\label和\ref

% 插入图片
% \begin{figure}[htbp!]
%     \centering
%     \includegraphics[width=6cm]{pic.jpg}\\
%     \caption{标题}\label{pic}
% \end{figure}

\cite{Otto_Forster_Riem_surfaces}
% \nocite{*}%星号表示引用bib文件中的全部文献
\bibliographystyle{plain}
\bibliography{reference}
\end{document}